\chapter{Verwendete Frameworks\label{cha:context}}
Bei der Überlegung, welche Frameworks für das Projekt verwendet werden sollten, wurden primär zwei in Betracht gezogen:
\begin{itemize}
\item Greensock
\item ScrollMagic
\end{itemize}

Die genutzten Frameworks sollten bestimmte Funktionen für die Umsetzung der Parallax-Scrolling Webseite ermöglichen. So war zunächst eine gute, umfangreiche API Dokumentation wichtig, mit der sich alle Gruppenmitglieder auseinandersetzen können. Ebenso sollten sie eine gute Performance ermöglichen und kompatibel mit den gängigen Browsern sein. \\
Horizontales, sowie vertikales Scrollen musste möglich sein und ebenso Funktionen für einen Zoom-Effekt. Durch die Einbindung der Zeichnungen war eine Unterstützung des svg-Formats von Nöten.
Schließlich fiel die Entscheidung sowohl Greensock als auch ScrollMagic zu verwenden, zusätzlich zur geläufigen Library jQuery.

	\section{Greensock}
Greensock erweitert das ebenso verwendete Framework ScrollMagic, bietet eine übersichtliche API-Dokumentation und ist wie gewünscht browserkompatibel. Die hauptsächlichste Verwendung ist „TweenLite“. Dieses ermöglicht die Animationen von Objekten, wodurch Dauer, Art, Delay und Ähnliches einer Animation bestimmt werden können. \\
Überzeugt hat auch, dass Greensock „advanced sequencing“ untersützt – die Überlappung von mehreren Animationen ineinander. \\
Negativ ist lediglich aufgefallen, dass es eine käuflich zu erwerbende Version von Greensock gibt, die einige Funktionen besser und einfacher darbietet. Die kostenlose Version hat jedoch ausreichend genügt.
	\newpage
	
	\section{ScrollMagic}
Eine Besonderheit des Frameworks ScrollMagic ist, dass es mobil kompatibel ist. Dies war zwar für die Wahl kein gesondert gewichtetes Kriterium, ist aber in Bezug auf Responsive Webdesign, welches ScrollMagic ebenfalls unterstützt, praktisch. Es bietet eine ausführliche API-Dokumentation und arbeitet ohne größere Probleme mit Greensock zusammen, was ebenso überzeugt, wie die Objektorientierheit des Frameworks. Diese vereinfacht den Umgang durch die Vorkenntnise aus den Modulen „OOP1 und OOP2“.
	
	\section{jQuery}
Die Library jQuery bietet zunächst die gewünschte Browserunabhängigkeit. jQuery ist kompakter und komfortabler als JavaScript, da es viele JavaScript Funktionen vereinfacht, auch dadurch, dass der Zugriff auf DOM-Elemente simpel ist. Ebenso lässt sich mit der Bibliothek die geforderten Fade-In/Out, Zoom- und Slide Effekte realisieren. Des Weiteren ist jQuery durch eine zusätzliche Datei auch von Greensock steuerbar, welches wir wie zuvor erwähnt parallel verwendet haben, und bietet eine einfache Zusammenarbeit mit anderen Frameworks.