\section{Implementierung der Auswahlmöglichkeit}
Um die Website noch lebhafter zu gestalten und die Story dem Nutzer näher zu bringen, haben wir uns dazu entschieden, Auswahlmöglichkeiten zu implementieren. Hier kann der Nutzer zwischen den Rollen wählen, die Adam angeboten bekommt und somit seinen Weg und seine Weltanschauung verändern. Es sind zwei mögliche Pfade implementiert, die man auf „gut“ und „böse“ herunterbrechen kann.
Umgesetzt wurde dies durch ein Haupt-div, das als Wrapper fungiert und das linksbündig die linke Szene, rechtsbündig die rechte Szene und mittig die mittlere Szene enthält. Dieses ist standardmäßig um die Breite einer Szene nach links geschoben, damit die mittlere Szene, die den Auswahlbildschirm darstellt, zuerst sichtbar ist.\\
Scrollen auf der x-Achse wird hier mittels overflowX: Hidden gelocked, also unterbunden.
Ein Klick auf einen der Auswahlbuttons startet dann eine Funktion, die den Scrollock aufhebt und per Greensock das Haupt-div entweder links oder rechts animiert, also verschiebt. Gleichzeitig wird der Counter für „good“ und „bad“ um 1 erhöht, um am Ende ausgezählt zu werden und die Schlussszene zu generieren. Um die Performance der Website zu steigern, springt die Timeline der nicht sichtbaren Szene direkt an das Ende.