\section{Implementierung der Szenen}
Die Vorgehensweise bei der Umsetzung der Story war durch eben diese so gut wie gegeben. Die einzelnen Szenen werden getrennt voneinander dargestellt und mittels Scrollen und fade In/Out-Effekten miteinander verbunden.\\
So konnte den Gruppenmitgliedern verschiedene Szenen zugewiesen werden, die bis zum Punkt der Verkettung und Abstimmung der Scrollgeschwindigkeit gesondert voneinander bearbeitet werden konnten.
Vorteilhaft war hierbei, dass schnell ein Grundgerüst erstellt werden konnte, dass lediglich angepasst werden musste, größtenteils mit IDs. Ebenso mit Klassen, die nach den ersten Szenen feststanden und weiterhin verwendet werden konnten. Dinge auszuprobieren war nur in einem begrenzten Rahmen möglich, da durch die CSS-Dateien die Gefahr bestand, andere Szenen ungewollt mit zu beeinflussen. Dopplung von Klassen konnten aber so vermieden werden. \\
Da die Szenen in Einzelarbeit erstellt wurden, war eine ausreichende Kommunikation vonnöten, um zu gewährleisten, dass Szenen nicht „kreuz und quer“, sondern in einer recht nahen Reihenfolge bearbeitet wurden. Ebenso musste mitgeteilt werden, wer sich welche Szene vornahm und an welcher Stelle vielleicht jemand durch Probleme übernehmen musste. Probleme hiermit konnten anhand von WhatsApp und Wire immer ausreichend schnell gelöst werden. 
