\section{Kommunikation}
Eines der wesentlichsten Probleme bei der Umsetzung des Projektes war die Kommunikation innerhalb der Gruppe. Dieses Problem zeigte sich vor allem darin, dass die Aufgabenverteilung für alle nicht immer ganz klar formuliert war beziehungsweise von einem Teil der zuständigen Leiter der Aufgabenbereiche eine gewisse Vorarbeit geleistet werden musste, bevor ein anderer übernehmen und damit weiterarbeiten beziehungsweise anfangen konnte. Durch eine verbesserte und stetige Kommunikation war dies aber ein kleineres Problem und keinesfalls ein Hindernis in der Vollendung des Projektes. 
Zur Lösung dieses Makels und zur gemeinsamen online Arbeit wurden verschiedene Plattformen ausprobiert. So konnten die Teammitglieder sich gegenseitig unterstützen, da jeder unterschiedliche Vorkenntnisse mit in die Projektgruppe brachte.
\begin{itemize}
\item Zu Anfang der Gruppenarbeit wurde eine Gruppe auf WhatsApp eingerichtet.
\item Zusätzlich dazu wurde GitLab zum Hochladen von Daten verwendet.
\item Da komplexe Fragen innerhalb der verschiedenen spezialisierten Gruppen (Story, Coding, Fotos) in der WhatsApp Gruppe nicht gut plastisch beantwortet werden konnten, wurde sich noch nach alternativen Plattformen umgeschaut.
\item Zuerst wurde das Programm Discord getestet, um sich dort via Sprachchat unterhalten zu können, um spezifisch auf bestimmte Fragen eingehen zu können. Jedoch wurde diese Plattform wegen nicht ausreichender Qualität, vor allem bei der Bildschirmübertragung, nicht weiterverwendet.
\item Als weitere Alternative wurde schließlich die Kommunikationssoftware Wire verwendet, welches die gleichen Funktionalitäten wie Discord zur Verfügung stellt, jedoch eine bessere Qualität der Übertragung aufwies. 
\item Zusätzlich wurde sich bei Bedarf in der Hochschule getroffen, um das weitere Vorgehen zu besprechen, gemeinsam zu arebeiten oder eventuelle Schwierigkeiten zu beseitigen.
\end{itemize}