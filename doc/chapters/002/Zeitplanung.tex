\section{Zeitplanung}
Am Anfang der dritten Phase des Projektes wurden drei Meilensteine und ein Datum festgelegt, bis zu welchem diese erreicht werden sollen. Bei genügend verbleibender Zeit bis zur Projektabgabe, wurde über sogenannte „Nice to have‘s“ entschieden, die nach der Erledigung der Meilensteinen noch implementiert werden können. Ebenso wurde darüber entschieden, dass nach anfänglichen Startschwierigkeiten bei der Implementierung, mindestens vier Szenen pro Woche geschafft werden müssen, um die Meilensteine rechtzeitig zu erreichen. Weitere Deadlines wurden intern über Whatsapp, Wire und Gruppenbesprechungen in der Hochschule festgelegt, dessen Einhaltung und Umsetzung stets konstant kommuniziert wurden. 
Zur Erinnerung wurden größere Deadlines zur Erinnerung im Board des GitLab Projektes eingetragen. So hatte jedes Gruppenmitglied jederzeit Zugriff darauf.\newpage Das GitLab-Projekt wurde von Verena verwaltet. Sie erstellte ein Repository, Boards, Issues und Tags, die die anderen Gruppenmitglieder änderten, wenn jeweils eine Aufgabe abgeschlossen wurde. Ebenso lehrte sie uns gemeinsam mit ihrem Mann, wie man Git verwendet.