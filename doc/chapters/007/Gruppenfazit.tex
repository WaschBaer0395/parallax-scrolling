\section{Fazit der Gruppenmitglieder}
\underline{Verena Gebauer}
Jeder aus unserer Gruppe brachte unterschiedliche Stärken und Vorkenntnisse mit, die wir bei der Aufgabenverteilung natürlich mit einbezogen haben. So haben wir uns optimal ergänzt und unser Wissen miteinander geteilt. \\
Da ich durch meine Tätigkeit bei der Valtech bereits in einem internen Projekt aktiv mitarbeiten konnte, sammelte ich bereits vor dem Informatikprojekt 1 erste Erfahrungen in der Projektorganisation und der Arbeit mit Git.\\
Diese Erfahrungen konnte ich in unser Projekt mit einfließen lassen und an meine Kommilitonen weitergeben.
Aufgrund meiner Vorkenntnisse war die Projektorganisation in GitLab eine meiner Aufgaben. Des Weiteren habe ich einen Großteil der Szenen implementiert. Beides hat mir sehr viel Spaß gemacht. Auch die Arbeit in der Gruppe hat mir sehr gefallen. \\
An einigen Stellen hätte die Kommunikation besser sein können, das hätte uns einiges an Zeit gespart. Auch die Fristen für die einzelnen Teilaufgaben hätten wir genauer definieren und festhalten können.
Für das erste Projekt hat die Gruppenarbeit aber alles in Allem recht gut funktioniert.
Ich hatte mich für das Projekt Parallax Scrolling beworben, da ich die gewonnenen Kenntnisse des Fachs Webprogrammierung anwenden und weiterentwickeln wollte. \\
Dieses Ziel habe ich erreicht. Ich bin viel sicherer in der Anwendung von CSS und Javascript geworden. Außerdem habe ich noch weitere Kenntnisse, zum Beispiel aus dem Bereich der Fotografie und Bildbearbeitung mitnehmen können. Das Projekt war, durch unsere etwas andere Umsetzung als die bisherigen Parallax-Scrolling-Projekte, sehr komplex. Auf diese Weise konnte man über die Webentwicklung hinaus einiges mitnehmen. Ich bin überzeugt, dass ich die nun gesammelten Erfahrungen wiederum in die Valtech mitnehmen kann. Ich habe meine Entscheidung für dieses Projekt in keiner Minute bereut.\\
\\
\\
\underline{Rene Hippe}
Für unser erstes, und auch mein erstes Projekt im Rahmen der Hochschule, muss ich sagen, dass ich sehr zufrieden mit unserem Ergebnis bin. Wir hätten eindeutig mehr Deadlines setzen, und uns weniger auf die nice to have’s konzentrieren sollen, jedoch hat es alles in allem dann doch gut geklappt.\\
Ich persönlich hatte mit Mathias Bethge zusammen das Grundgerüst und die Frameworks Greensock und Scrollmagic als Aufgabe. Wir haben die Animationen und die Struktur der Javascript für die einzelnen Szenen festgelegt und so dann auch den anderen Gruppenmitgliedern erklärt wie man Szenen erstellt und Animationen auf diesen Szenen ausführt.\\
Durch das Projekt habe ich definitiv viel über den Umgang mit Frameworks gelernt und vor allem dem Umgang mit Greensock und Scrollmagic, sowie deren Möglichkeiten und Vorteile gegenüber herkömmlichen, selbstgeschriebenen Funktionen für Animationen etc. \\
Die Teamarbeit hat auch bis auf eine Ausnahme gut funktioniert. Wir haben uns über Wire in einer Konferenzschaltung verständigt und zusammen Ziele erarbeitet, falls wir nicht gerade in der Hochschule zusammensaßen oder uns woanders getroffen haben. \\
Alles im allen bin ich, wie schon am Anfang erwähnt, sehr zufrieden mit unserem ersten Projekt.
Ebenso kann ich mir auch eine weitere Zusammenarbeit für spätere Projekte (auch in anderen Bereichen der Medieninformatik) gut vorstellen.\\
\\
\\
\underline{Nadia Radau}
Mit unserem Projektergebnis bin ich sehr zufrieden. Am Ende dieser Phase zu sehen, dass sich all die Zeit, Mühe und Herzblut gelohnt haben, hinterlässt ein positives Gefühl in Hinblick auf die komplette Projektarbeit. Es war unfassbar lehrreich sich mit einem Team auseinanderzusetzen, dass die gleichen Ziele verfolgte und ebenso viel reinsteckte wie man selbst. Wo Hilfe gebraucht wurde, bekam man sie. Raum für zielführende Diskussionen waren immer gegeben, dabei wurde auf konstruktive Ehrlichkeit nicht verzichtet, was ich als sehr förderlich empfand. Ebenso war es interessant zu erforschen, was alles umsetzbar war von unseren Ideen und vor allem mit unserem Vorwissen und dem, was wir uns schließlich noch aneignen mussten. Erfrischend war, dass wir nicht an die Hand genommen wurden und somit gezwungen waren, eigenverantwortlich zu arbeiten und dabei nicht zu vergessen, dass mehr Noten als die eigene in den Händen liegen. Teils beängstigend, aber eigentlich immer eine Freude, wenn wieder gemeinsam an einem Problem gearbeitet wurde. Ein Lächeln kam dabei immer zu Stande.\\
Zeitmanagement war bei uns nicht sonderlich großgeschrieben, hat uns aber eher noch mehr Freiheit gegeben und Druck genommen, der vielleicht Gegenteiliges bewirkt hätte.\\
Schlussendlich kam nie ein „Nein!“, falls etwas angesprochen wurde, das nicht optimal lief. Im Gegenteil, es wurde immer eine Lösung gesucht und versucht, alle Gruppenmitglieder vom weiteren Vorgehen zu überzeugen.\\
Ich danke allein dafür meiner Gruppe, mich gelehrt zu haben, dass so eine Zusammenkunft gut funktionieren kann und man mit einem erfolgreichen Projekt endet.\\
Ebenso war es interessant, zusammenhängende Texte zu einem Thema beziehungsweise zu einer Story zu schreiben. Daran hatte ich mich vorher nicht getraut, abgeschreckt von möglicher Schreibblockade.
Insgesamt hat mir das Projekt sehr gut gefallen und würde jedem empfehlen, sich darauf zu bewerben.\\
\\
\\
\underline{Mathias Bethge}
Zu Beginn des Projektes habe ich mich mit Rene gemeinsam in Greensock eingearbeitet und wir haben versucht einen gemeinsamen Nenner bezüglich des Aufbaus des Codes der Animationen zu finden und uns generell mit den Methoden des Frameworks auseinandergesetzt. Zum Schluss haben wir unser Erarbeitetes übersichtlich in den Javadateien dokumentiert und es den anderen Gruppenmitgliedern per Wire erklärt, damit wir alle auf demselben Stand sind.\\
Im nächsten Schritt, und als die Fotos des ‚Adam‘ fertig geschossen waren, und gemeinsam mit der Gruppe ausgewählt worden sind, habe ich mich primär an die Erstellung der einzelnen Szenen gesetzt. Dies entpuppte sich als aufwändiger als gedacht, da wir ein gewisses Maß an Qualität haben wollten und alle Bilder komplett bearbeitet werden mussten, damit sich alle hinzugefügten Personen gut in die Szenen einfügen konnten und niemand fehl am Platze erschien.\\
Die Gruppenarbeit, sowie -Kommunikation war eine sehr gute, zuverlässige und angenehme. Da es jedoch zu einer kleinen Ausnahme kam, habe ich neben der Bildbearbeitung, und Zeichnungserstellung, ebenfalls geholfen die Szenen z.B. Animationen der Szenen zu implementieren.\\
Neben der Bildbearbeitung lag mein Aufgabenbereich ebenfalls in der Erstellung der Zeichnungen, welche ‚Adam‘ auszeichnen sollten. Diese wurden von mir in der Webseite implementiert und animiert, so dass sich niemand anderes darum zu kümmern brauchte.\\
Abschließend kann ich sagen, dass mir dieses Projekt sehr viel Spaß gemacht hat, und ich zudem einen besseren Einblick in JavaScript bekommen habe, auch, wenn wir ein Framework benutzt haben, in welchem wir lediglich Methoden aufgerufen haben.\\
Die zeitlichen Meilensteine beziehungsweise kleinere Schritte, hätten wir definitiv besser planen können. So haben wir unser eigentlich vorgegebenes Pensum von 3 komplett fertig implementierten Szenen pro Woche nicht eingehalten und lediglich immer mal wieder am Projekt weitergearbeitet. Dies hat schlussendlich funktioniert, hätte uns aber auch am Ende im Zweifel das Projekt kosten können. Oder einige Nächte.
Auch hat die gesamte Gruppe, denke ich, eine Menge bezüglich zeitlicher Planung, Organisation eines recht großen Projektes, sowie Kommunikation, auch außerhalb der Komfortzone, gelernt.\\
\\
\\
\underline{Aleyna Yildirimhan}
Die Zusammenarbeit im Team hat Spaß gemacht, aber war auch relativ anstrengend. Ich habe gerne am Projekt mitgearbeitet und finde, dass alle Teammitglieder äußerst viel Zeit und Fleiß in das Projekt investiert haben. Die Vorkenntnisse einiger Gruppenmitglieder sind natürlich von Vorteil gewesen. Sei es in Photoshop oder in der Einarbeitung des Frameworks, waren alle Beteiligten stets bereit sich gegenseitig zu helfen. \\
Das Arbeiten über Git-Lab hat uns vieles vereinfacht. Durch das Board und die Struktur waren die Aufgaben viel übersichtlicher und verständlicher zu verfolgen. Jedes Teammitglied hatte den Zugriff auf den aktuellsten Stand des Projekts, und konnte auch ohne das eigentliche Projekt zu beeinflussen am Code arbeiten. Das Prinzip von Git-Lab ist natürlich ganz praktisch für so eine Projektarbeit. \newpage
Ich finde in den vier Monaten haben wir als Team eine gute Arbeit geleistet. Es gab teilweise auch Tage, an denen wir stundenlang an einer Szene gearbeitet haben, damit sie unseren Vorstellungen entspricht. Dabei sind natürlich auch Momente gewesen, in denen wir uns nicht alle einig waren, dem entsprechend auch einige Auseinandersetzungen hatten. Aber solche Auseinandersetzungen sind nun mal in einer aufwändigeren Gruppenarbeit vorhanden und nicht zu vermeiden. Trotz dessen haben wir uns im Team gegenseitig bei den Aufgaben stets unterstützt und ununterbrochen weitergearbeitet. \\
Im Großen und Ganzen war das eine etwas anstrengende Zeitspanne, aber dennoch eine sehr Lehrreiche. Ich bedanke mich bei allen Teammitgliedern für die Bereitschaft, den Einsatz und den Fleiß.