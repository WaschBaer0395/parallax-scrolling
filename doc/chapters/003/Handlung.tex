\section{Handlung}
Im Fokus der Story steht Adam Candid, eine menschenähnliche künstliche Intelligenz, welcher von einer Senior UX Designerin in die Schauspiel-Szene eingeschleust wird. Daraufhin wird Adam bei seinen Dreharbeiten für verschiedene Filme verfolgt.\\
\\
\begin{itemize}
\item Zunächst wird Quinn Walker nach dem Vorschlag eines Richtungswechsels der Firma Lacuna Industries, für die sie arbeitet, abgewiesen.
\item Hiernach setzt sie die Idee eines humanoiden Roboters dennoch heimlich um.
\item Nach dem Bau ihres Roboters schleust sie diesen in die Schauspiel-Szene ein, um zu beweisen, dass er sich nicht merklich von einem Menschen unterscheidet und eine Angst demnach unbegründet ist.
\item Adam ergattert seine erste, große Rolle in einem Film über eine Beziehung eines Arztes und seiner totkranken Patientin.
\item Schließlich stellt sich Adam seinem ersten Interview mit der Presse. Die Medien werden auf ihn aufmerksam.
\item Je nach Genre des Films lernt Adam andere Dinge, dies verändert die Weltanschauung der KI.\newpage
\item Erste Möglichkeit der Entscheidung: 
\begin{itemize}
\item Adam landet eine weitere Rolle, bei der er einen Entführer spielt, der die Tochter eines korrupten Politikers entführt.
\item Adam spielt einen Ehemann, dessen Frau ihn aufgrund seiner Spielsucht verlässt und der daraufhin eine Selbstfindungsreise in die Berge unternimmt.
\end{itemize}
\item Adam ist auf seine erste große Gala eingeladen und bekommt einen Preis als bester Newcomer. Die Leute reden darüber, dass er keine Freundin hat und nie auffällig wird.
\item Zweite Möglichkeit:
\begin{itemize}
\item Adam spielt einen Serienkiller in einem Crime-Thriller, der mit Vorlage einer Buchreihe Menschen tötet.
\item Adam spielt einen alleinerziehenden Vater, dessen Sohn an Autismus leidet, er zeigt ihm eine andere Welt und beide lernen voneinander.
\end{itemize}
\item Quinn engagiert eine Frau von einem sehr diskreten Escort-Service, die ein paar Dates mit Adam in der Öffentlichkeit haben soll, damit sich die Presse beruhigt.
\item Letzte Auswahlmöglichkeit:
\begin{itemize}
\item Adam spielt einen jungen Schriftsteller, der am Weihnachtsmorgen am Flughafen festhängt und sich in eine Frau verliebt, die Weihnachten hasst und ihn mit zu ihrer Familie nimmt.
\item Adam spielt einen Stalker, der sich in eine Sängerin verliebt und nach und nach andere Fans von ihr umbringt, bis er ihr endlich gegenübersteht.
\end{itemize}
\item Schlussszene wird aufgrund der vorherigen Auswahlmöglichkeit entschieden:
\begin{itemize}
\item Adam bekommt einen Oscar in der Kategorie „Bester Hauptdarsteller“ verliehen. Quinn nutzt diese Chance um der Welt zu offenbaren, dass Adam eigentlich eine hoch entwickelte künstliche Intelligenz ist und offensichtlich nicht von einem Menschen zu unterscheiden. Daraufhin hält einen positiven Monolog über die Menschheit.
\item Auf dieselbe, zuvor Beschriebene Situation hin hält Adam einen negativen Monolog über die Menschheit.
\end{itemize}
\end{itemize}